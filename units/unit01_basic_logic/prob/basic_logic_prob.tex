\documentclass[11pt]{article}

\usepackage{fullpage}
\usepackage{amsmath, amssymb, bm, cite, epsfig, psfrag}
\usepackage{graphicx}
\usepackage{float}
\usepackage{amsthm}
\usepackage{amsfonts}
\usepackage{cite}
\usepackage{hyperref}
\usepackage{pgf,tikz}
\usepackage{enumitem}
\usepackage{mathtools}
\usepackage{siunitx}
\usepackage{../../styles/course}


\begin{document}

\title{Introduction to Hardware Design\\
Basic Digital Logic:  Solutions}
\author{Profs. Sundeep Rangan, Siddharth Garg}
\date{}

\maketitle

\begin{enumerate}
\item \emph{Propagation delay}:  A signal $x$ has time-value change points
\[
    (10, 0), (20, 4), (30, 2)
\]
where the time is in nanoseconds.  The signal is passed through
a circuit that produces $y = 2x+1$ with a propagation delay of \SI{3}{ns}.
What are the time-value change points of $y$?  Use $\text{x}$ for unknown values.

\item \emph{Critical path}: Consider the following SystemVerilog code:
\begin{systemverilogcode}
  always_ff @(posedge clk) begin
      x1 <= 3*x1>>2 - 5*x>>3 + 7;
      x2 <= 3*x1>>2 +  2;
  end
\end{systemverilogcode}
Here \texttt{3*x>>2} denotes multiplication by 3 followed by a right shift
by 2 bits (division by 4).  
Assume that:
\begin{itemize}
\item Each multiplication with shift takes \SI{4}{ns}.
\item Each addition or subtraction takes \SI{2}{ns}.
\item Registers have zero delay.
\end{itemize}
What is the critical path delay of this circuit?


\item \emph{Sequential updates}:
Consider the following SystemVerilog code:
\begin{systemverilogcode}
  always_ff @(posedge clk) begin
      x <= x + v;
      if (x > 30) begin
          v <= -10;
      end else if (x < 0) begin
          v <= 10;
      end
  end
\end{systemverilogcode}
Starting from $(x,v)=(15,10)$, what are the values of $(x,v)$
for the next 5 clock cycles?

\item \emph{Splitting paths over multiple cycles}: Consider the following
SystemVerilog code:
\begin{systemverilogcode}
  always_comb begin
      act_in = w1*xreg+b1;
      if (act_in > 0) begin
          a = act_in;
      end else begin
          a = 0;
      end
      xsq = xreg*xreg>>4;
      y = xsq + w2*a + b2;
  end
  always_ff @(posedge clk) begin
      xreg <= x;
  end
\end{systemverilogcode} 
where \texttt{>>4} denotes a right shift by 4 bits (division by 16).
So, the code registers the input \texttt{x} into \texttt{xreg} on each clock cycle,
and produces the output \texttt{y} in a single clock cycle based on the registered value.
The circuit is to be syntheized with components:
\begin{itemize}
\item Multiplication with shift:  \SI{4}{ns}
\item Addition or subtraction:  \SI{2}{ns}
\item Logic to compute $\max\{0,u\}$:  \SI{1}{ns}
\item Registers:  zero delay
\end{itemize}
Assume that $w_1$, $b_1$, $w_2$, and $b_2$ are constants.

\begin{enumerate}[label=(\alph*)]
\item Draw a block diagram of the circuit showing an implementation of the circuit.
If this is being auto-graded, you do not need to submit the diagram.  But, I will include it in the solutions.

\item What is the critical path delay of this circuit?
\item Rewrite the code to operate over two clock cycles instead of one
to minimize the critical path delay.
What is the critical path delay of the new circuit?
\end{enumerate}

\item \emph{Exponent}
Suppose we wish to implement the function
\[\
    y = x^i, 
\]
with an integer exponent $i \in \{0,1,2,3\}$.
The input $x$ and output $y$ are signed short integers
-- do not worry about overflow.
Write a SystemVerilog module that computes $y$.
The output should always be 2 cycles after the input, 
even if $i=0,1$ or $2$.  Use only one multiplication in 
each clock cycle.  

Hint:  This problem is a bit harder.
You will need to use delay lines to store
the input $x$ and exponent $i$.


\end{enumerate}

  \end{document}

